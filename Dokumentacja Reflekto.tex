\documentclass[a4paper,11pt]{article}
%Przydatne paczki:
\usepackage{amssymb}
\usepackage{amsthm}
\usepackage{amsmath}
\usepackage[colorinlistoftodos]{todonotes}
\usepackage[colorlinks=true, allcolors=blue]{hyperref}
%Definicja kodowania i języka:
\usepackage[polish]{babel}
\usepackage[MeX]{polski}
\usepackage[utf8]{inputenc}
\usepackage[T1]{fontenc}
%Paczki dodane w drodze pisania:
\usepackage{graphicx}
\usepackage{anysize}
\selectlanguage{polish}
\usepackage{tabularx}
\usepackage[export]{adjustbox}
\usepackage{listings}
\usepackage{float}
\usepackage{fancyhdr}

%Nagłówek:
\pagestyle{fancy}
\fancyhead{}
\fancyhead[L]{\small{\bfseries \thepage}}
\fancyfoot[L, C, R] {}
\fancyhead[C]{\small{\bfseries Dokumentacja projektu "Reflekto"}}
\renewcommand{\headrulewidth}{0.8pt}

%\marginsize{left}{right}{top}{bottom}
\marginsize{2.5cm}{2.5cm}{2.5cm}{2.5cm}

\begin{document}

\title{Dokumentacja projektu \\ \textbf{,,Reflekto'' } }
\author{Michał Kwiecień \\ Michał Wójcik}
%skomentować żeby nie było daty
%\date{\vspace{-5ex}}
\maketitle

\begin{abstract}
Dokumentacja projektu inteligentnego lustra w konwencji IoT komunikującego się ze smartfonem z użyciem interfejsu Bluetooth Low Energy. Projekt powstał na potrzeby konkursu Nordic Semiconductor Student Contest. 
\end{abstract}

\begin{figure}
	\includegraphics[width=0.3\textwidth,center]{logo_nordic.png}
\end{figure}

\cleardoublepage
\tableofcontents
\clearpage

\section{Wstęp}



	
\end{document}